%%=============================================================================
%% Inleiding
%%=============================================================================

\chapter{Inleiding}
\label{ch:inleiding}
In de inleiding zal er een introductie volgen tot enkele belangrijke termen die meermaals zullen terugkeren in het onderzoek. Hier zal er ook een korte geschiedenis meegegeven worden over de content management systemen dat gekozen werden voor het onderzoeken. Na dit hoofdstuk zal het duidelijk zijn waarom deze studie van belang is en voor wie deze bedoeld is.

\section{E-commerce}
\label{sec:E-commerce}
\subsection{E-Commerce}
E-commerce, vaak ook wel E-Business genoemd, is de verzameling van alle transacties die op een digitale manier gebeuren. Het bekendste voorbeeld hiervan is natuurlijk, het aankopen van een artikel op een online webshop. De term e-commerce dateert van de jaren '70, hier werd deze gebruikt voor het beschrijven van het systeem dat de banken gebruikt voor het onderling uitwisselen van geld. De laatste jaren is deze term razend populair geworden door de enorme stijging in het online winkelen. Ondersteunde activiteiten worden vaak ook als e-commerce gezien. Denk maar aan zaken zoals het afhandelen van online betalingen, leveren van de goederen en digitale marketing. \autocite{MarketingTermen2018}

\subsection{Content Management Systeem}
Content Management Systeem, of afgekort CMS, is een systeem dat achter de website zit. Dit systeem stelt jou in staat om makkelijk jouw website te beheren. Deze systemenen zijn vaak voorzien van een eenvoudige interface waar je zonder enige kennis van Hypertext Markup Language (HTML), Cascading Style Sheets (CSS) en programmeertalen. De beheerder kan hierdoor eenvoudig content toevoegen en verwijderen. De toevoegingen gebeuren meestal aan de hand van een WYSIWYG-editor(what you see is what you get),zo een editor komt veel mensen bekend voor, gezien de vele gelijkenissen met het populaire tekstverwerking programma Word. \autocite{forresult2014,wphandleiding2015}


Uit de statistieken van w3techs blijkt dat op 54\% van alle websites op het internet een Content Management Systeem draait.  Van die 54\% draait 32.9\% op Wordpress wat dus resulteert in een marktaandeel van 59,6\%. Joomla draait op 3.0\% van de 54\% wat resulteert in een marktaandeel van 5.6\%. Drupal draait op 1.9\% van de 54\% wat resulteert in een marktaandeel van 3.6\%. Uit deze statistieken kunnen we ook makkelijk afleiden dat Wordpress,Joomla en Drupal de drie populairste Content Management Systemen zijn. Logischerwijs zijn dit ook de content management systemen die in dit onderzoek onderzocht worden. \autocite{w3techs2018}


\subsection{Wordpress}
Wordpress is een open-source content management systeem, open-source wilt zeggen dat iedereen toegang heeft tot de broncode en hier aan kan bijdragen. Het avontuur van Wordpress begon in 2003 met Matt Mullenweg en Mike Little. Het startte toen als een fork van het populaire blogsysteem b2/cafelog. Toen er plots geen updates meer kwamen van het populaire blogsysteem lanceerde Matt het idee om een nieuw systeem te ontwikkelen op basis van b2/cafelog. Mike hapte onmiddellijk toe en zo startten ze samen het avontuur dat nu Wordpress heet. Wat startte als een blogsysteem is ondertussen uitgegroeid tot een volwaardig CMS met een heel ecosysteem aan plug-ins en themes.\autocite{Postma2018}

\subsection{Drupal}
Net zoals Wordpress is Drupal ook een open-source content management systeem. In 2000 lanceert de Antwerpse universiteit student Dries Buytaert een kleine content framework. Het framework draaide op het interne netwerk dat hij en medestudenten opgezet hadden tussen hun kamers. Het framework diende voor hun onderling communicatie zoals de status van het netwerk dat ze opgezet hadden, waar ze de volgende dag zouden eten,etc. Nadat Dries afstudeerde, lanceerde hij het framework online zodat hij en zijn kameraden in contact konden blijven. Drop.org was geboren. Al snel trok het een hele andere publiek aan, er werden steeds gesproken over de nieuwste trends binnenin webtechnologieën, deze trends werden uiteindelijk methoden en kenmerken van de software waarop drop.org draaide. In janauri 2001 maakte hij de software achter drop.org open-source, nu kon iedereen vrij de software uitbreiden met hun nieuwste ideeën. Hij noemde de software Drupal.\autocite{Drupal2018}

\subsection{Joomla}
Joomla is net zoals Wordpress en Drupal ook een open-source content management systeem. In 2001 brengt het Australische bedrijf Miro het open-source CMS Mambo uit. Mambo was vlak na de release vrij populair, maar na een meningsverschil tussen het Mambo core team en het besturingscomitée, nam het grootste deel van het mambo core team ontslag. Dit resulteerde in de entiteit Open Source Matters en een code fork van Mambo genaamd Joomla. In 2005 werd de eerste versie van Joomla gelanceerd, dit was voornamelijk rebranding en enkel updates in vergelijking met het originele Mambo. Ondertussen is Joomla uitgegroeid tot een internationaal erkend content management systeem, Mambo wordt enkel nog vermeld om de geschiedenis van Joomla te beschrijven.\autocite{Crowder2009} 


\section{Onderzoeksvraag}
\label{sec:onderzoeksvraag}

Wees zo concreet mogelijk bij het formuleren van je onderzoeksvraag. Een onderzoeksvraag is trouwens iets waar nog niemand op dit moment een antwoord heeft (voor zover je kan nagaan). Het opzoeken van bestaande informatie (bv. ``welke tools bestaan er voor deze toepassing?'') is dus geen onderzoeksvraag. Je kan de onderzoeksvraag verder specificeren in deelvragen. Bv.~als je onderzoek gaat over performantiemetingen, dan 

\section{Onderzoeksdoelstelling}
\label{sec:onderzoeksdoelstelling}

Wat is het beoogde resultaat van je bachelorproef? Wat zijn de criteria voor succes? Beschrijf die zo concreet mogelijk.

\section{Opzet van deze bachelorproef}
\label{sec:opzet-bachelorproef}

% Het is gebruikelijk aan het einde van de inleiding een overzicht te
% geven van de opbouw van de rest van de tekst. Deze sectie bevat al een aanzet
% die je kan aanvullen/aanpassen in functie van je eigen tekst.

De rest van deze bachelorproef is als volgt opgebouwd:

In Hoofdstuk~\ref{ch:stand-van-zaken} wordt een overzicht gegeven van de stand van zaken binnen het onderzoeksdomein, op basis van een literatuurstudie.

In Hoofdstuk~\ref{ch:methodologie} wordt de methodologie toegelicht en worden de gebruikte onderzoekstechnieken besproken om een antwoord te kunnen formuleren op de onderzoeksvragen.

% TODO: Vul hier aan voor je eigen hoofstukken, één of twee zinnen per hoofdstuk

In Hoofdstuk~\ref{ch:conclusie}, tenslotte, wordt de conclusie gegeven en een antwoord geformuleerd op de onderzoeksvragen. Daarbij wordt ook een aanzet gegeven voor toekomstig onderzoek binnen dit domein.

