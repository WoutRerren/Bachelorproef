\chapter{Stand van zaken}
\label{ch:stand-van-zaken}

% Tip: Begin elk hoofdstuk met een paragraaf inleiding die beschrijft hoe
% dit hoofdstuk past binnen het geheel van de bachelorproef. Geef in het
% bijzonder aan wat de link is met het vorige en volgende hoofdstuk.

% Pas na deze inleidende paragraaf komt de eerste sectiehoofding.
Content management systemen bestaan al sinds het begin van de jaren 2000, deze systemen zijn dus geen overnight sensation en bestaan al meer dan 10 jaar. Natuurlijk zijn er in al die tijd al meerdere onderzoeken gevoerd hiernaar. Dit zijn ver uiteenlopende onderzoeken van algemene vergelijking tussen de content management systemen tot de security van de third party plugins. 

In het artikel van \textcite{Patel2011} wordt er een algemene vergelijking gemaakt tussen Joomla, Wordpress en Drupal. Door het populair gegeven dat content management systemen nu eenmaal is, is er zoveel keuze uit zoveel verschillende content management systemen.
\textcite{Patel2011} kozen hun content management systemen aan de hand van drie factoren, namelijk populariteit, documentatie en Google page rank (Google page rank bepaalt hoe belangrijk een webpagina is, het is een van de factoren die bepaalt welke webpagina's weergegeven worden in de Search Engine Results).Uit de bijgeplaatste grafiek in het artikel kan men duidelijk afleiden dat Joomla, Wordpress en Drupal steeds de top 3 vormen, vandaar de keuze voor deze systemen.

Vervolgens hebben ze deze 3 systemen onderling vergeleken op basis van volgende factoren: 3 verschillende populariteit vergelijkingen(aantal keer gezocht in Google, voorkeur CMS door de gebruiker, \% van populairste websites dat bepaald CMS draaien), gemiddelde kosten vergelijking, features vergelijking, Social bookmarking vergelijking. \textcite{Patel2011} concluderen en suggereren volgende zaken uit hun onderzoek: een klein bedrijf zonder complexe requirements gaat beter voor Joomla omwille van de lage kosten in vergelijking met de andere, als je een complexere, meertalige webshop nodig hebt kies je beter voor Drupal, Wordpress plaatsen ze tussen Joomla en Drupal door de vele veranderingen de voorbije jaren.

Het onderzoek van \textcite{Patel2011} is ondertussen 7 jaar geleden, nu zal men in deze studie trachtte achterhalen wat de evolutie van deze content management systemen was de voorbije jaren. Klopt het nog steeds dan Joomla de betere keuze is voor een kleine, niet complexe website? Is Drupal nog steeds de beste optie voor een meertalige, complexe webshop? Bevindt Wordpress zich nog altijd tussen Joomla of Drupal? In de conclusie van dit onderzoek zal dit zeker vermeld worden. Dit onderzoek onderscheidt zich van \textcite{Patel2011} hun onderzoek door andere succesfactoren te gebruiken en het effectief uitbouwen van basis websites met de gekozen content management systemen.


In de scriptie van \textcite{Crombrugge2015} wordt er expliciet gekeken naar Drupal en e-commerce als een geschikte combinatie. In dit onderzoek zijn de eerste hoofdstukken vooral een kennismaking. Je leert er meer over Drupal zoals: de geschiedenis, de structuur, de key features, waarom en wanneer Drupal en Drupal vs the others(Wordpress, Joomla, Django, Modx, Concrete5, Silverstripe).  Alsook leer je er kennis maken met e-commerce, zaken zoals de verschillende licentiemodellen, sales modellen en e-commerce oplossingen (Magento, OpenCart, Prestashop, WooCommerce, Shopify).In het onderzoek gedeelte van de scriptie van \textcite{Crombrugge2015} wordt er allereerst een aantal zaken beschreven zoals: beschikbare e-commerce modules, Drupal Commerce en Waarom Drupal commerce? Vervolgens volgt het eigenlijk onderzoek door middel van een concurrentie analyse van volgende e-commerce oplossingen (Magento, Ubercart, WooCommerce en Shopify). Hij geeft een beschrijving van de systemen en enkele van hun belangrijkste kenmerken, vervolgens legt hij uit waarom je best dat systeem gebruikt en ten slotte zal hij dit systeem dan vergelijken met Drupal commerce. 

In de conclusie van het onderzoek kan men vervolgens het antwoord lezen op zijn onderzoeksvraag: "“Ja”, Drupal is geschikt in combinatie met e-commerce, het is zelfs één van de beste e-commerce oplossingen, rekeninghoudend met het juiste type project."\parencite{Crombrugge2015} Met dit in gedacht zal men in dit onderzoek trachtte te achterhalen of Joomla en Wordpress hier ook bij horen alsook zal men dan kijken of dat Drupal de beste is of niet. Het grote verschil ook met het onderzoek van \textcite{Crombrugge2015} is dat er in dit onderzoek ook effectief een kleine webshop zal opgezet worden.

Over de veiligheid van deze content management systemen zijn er ook al verschillende onderzoeken gevoerd. In het onderzoek van \textcite{DeHaes2016} heeft men onderzoek gedaan naar de veiligheid van deze content management systemen a.d.h.v. de OWASP (Open Web Application Security Project) top 10. In dit onderzoek komt men tot de conclusie dat de standaard implementaties van deze systemen veilig zijn. Er zijn pas problemen met de veiligheid wanneer de gebruikers volgende zaken uitvoeren: contributie aan de broncode, third party plug-ins toevoegen, zelf ontwikkelde module toevoegen, misconfiguratie van het systeem. Een ontwikkelaar gaat deze bedreigingen tot de veiligheid al makkelijker opmerken en de nodige stappen voor ondernemen.

In dit onderzoek zal men trachtte te achterhalen of de nodige middelen beschikbaar zijn voor de niet-technische mensen om een veilige webapplicatie op te bouwen. Ten slotte speelt de veiligheid een kritisch rol in de keuze van het juiste content management systeem, als eindgebruiker wil je toch niet enkel eenvoudig een website opzetten maar ook een veilige website.