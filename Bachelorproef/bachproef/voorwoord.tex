%%=============================================================================
%% Voorwoord
%%=============================================================================

\chapter*{Woord vooraf}
\label{ch:voorwoord}

%% TODO:
%% Het voorwoord is het enige deel van de bachelorproef waar je vanuit je
%% eigen standpunt (``ik-vorm'') mag schrijven. Je kan hier bv. motiveren
%% waarom jij het onderwerp wil bespreken.
%% Vergeet ook niet te bedanken wie je geholpen/gesteund/... heeft

Het uitschrijven van een bachelorproef zag ik persoonlijk altijd als een unieke kans. Aangezien je de kans krijgt om een onderwerp waarin je geïnteresseerd in bent verder te onderzoeken en uit te werken. In het eerste semester van mijn derde jaar Toegepaste Informatica leerde ik kennis maken met het content management systeem Drupal. Dit was mijn eerste ervaring met een content management systeem, een interesse ontwikkelde zich. Ik begon mij af te vragen welke content management systemen er naast Drupal bestonden. Ik vroeg mij ook af welke gelijkenissen en verschillen je zou opmerken als je enkele content management systemen met mekaar zou vergelijken. Een bachelorproef leek mij dan ook de ideale opportuniteit om dit te onderzoeken.

Dit gedeelte zou ik graag gebruiken om verschillende mensen te bedanken die mij gesteund hebben tijdens het uitwerken van deze bachelorproef. Natuurlijk wil ik graag de leerkrachten van de Hogeschool Gent bedanken die mij gevormd hebben tot de IT 'er dat ik vandaag de dag ben. Door hun beschikte ik over de nodige kennis om deze bachelorproef tot een goed einde te brengen. Er is één lector in het bijzonder dat ik graag zou willen bedanken namelijk mevrouw Noemie Slaats. Door het geven van consistente, concrete feedback evolueerde dit onderzoek consistent tot een kwalitatief eindwerk.

Ook zou ik graag mijn co-promotor, Joris De Groot van calibrate, willen bedanken. Zijn inzichten en feedback als Drupal-expert draagde bij tot een verhoogde kwaliteit van mijn eindwerk.

Ook zou ik graag mijn familie en vrienden bedanken voor hun steun niet enkel tijdens het uitwerken van mijn bachelorproef, maar gedurende heel mijn opleiding. In het bijzonder zou ik graag mijn moeder, Françoise Beeken, bedanken. Door haar belangrijke rol in mijn opvoeding was ik voorzien van de nodige eigenschappen om deze opleiding tot een goed einde te brengen.

Ik heb ervoor gekozen om de technische woordenschat tot een minimum te behouden. Hierdoor kunnen mensen met zowel een technische als niet-technische achtergrond begrijpen welk content management systeem het best bij hun nood past.