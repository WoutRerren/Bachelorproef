%---------- Inleiding ---------------------------------------------------------

\section{Introductie} % The \section*{} command stops section numbering
\label{sec:introductie}

Een website is vandaag de dag niet meer weg te denken uit de wereld. Tegenwoordig heeft elk bedrijf een website, van eenmanszaken tot grote ondernemingen, zijn er miljoenen blogs en noem maar op. Hierdoor was er een enorme vraag naar makkelijk beheer van websites, dit had als gevolg dat er veel content management systemen ontstaan zijn de voorbije jaren. Door dit ruim aanbod is niet altijd even duidelijk welk CMS nu de beste optie is. Deze studie heeft als doel aan te duiden welk CMS best gebruikt kan worden voor een specifiek scenario. Aan de hand van volgende onderzoeksvragen gaan we antwoorden zoeken:
\begin{itemize}
	\item Welk CMS-systeem is optimaal voor het uitwerken van een volledig werkende E-commerce website?
	\item Welk CMS-systeem is optimaal voor het uitwerken van een volledig werkende Blog website?
	\item Welk CMS-systeem is optimaal voor het uitwerken van een volledig werkende SEO-optimized website?
	\item Welk CMS-systeem is optimaal voor het uitwerken van een volledig werkende one page website?
\end{itemize}

%---------- Stand van zaken ---------------------------------------------------

\section{State-of-the-art}
\label{sec:state-of-the-art}

In de bachelorproef van \textcite{VanCrombrugge2015-2016} wordt er uitgebreid aandacht besteed aan Drupal. Hij legt in zijn onderzoek al een eerste vergelijking tussen Drupal en andere populaire CMS'en. In dit onderzoek wordt vooral de focus op Drupal in het e-commerce scenario gelegd. De meest populaire e-commerce oplossingen worden opgesomd met hun sterke en zwakke punten. Deze opsomming zal aandachtig bestudeerd worden. Aan de hand van deze studie kan er beslist worden voor welke e-commerce oplossing best geopteerd wordt. Uit zijn onderzoek is er af te leiden dat de beste e-commerce oplossing afhankelijk is van de gebruiker zijn specifieke nood maar op het einde is Drupal commerce de beste oplossing. Logischerwijs zou het antwoord op de onderzoeksvraag in verband met e-commerce Drupal moeten zijn. Dit zal het onderzoek proberen aantonen.

%---------- Methodologie ------------------------------------------------------
\section{Methodologie}
\label{sec:methodologie}

Het beantwoorden van de onderzoeksvragen zal bereikt worden door het uitwerken van drie websites. Deze websites zullen uitgewerkt worden in drie verschillende CMS'en namelijk Drupal, Wordpress en Joomla. De keuze voor deze drie CMS'en is een verstaanbare keuze aangezien deze samen verantwoordelijk zijn voor 60 percent van de websites die gebruik maken van CMS technologie\autocite{BWPL2017}. De drie websites zijn één one-page SEO-geoptimaliseerde portfolio website over mijzelf , één volledig werkende e-commerce website en één volledig werkende blog-website. Voor dit onderzoek zal er ook een persoonlijke hosting aangekocht worden zodat de hosting factor van het CMS overal dezelfde is. Aan de hand van enkele factoren zal er dan bepaald worden welk CMS hiervoor de beste optie is.
Deze factoren zijn :
\begin{itemize}
	\item Multiplatform ondersteuning
	\item Responsiveness
	\item Gebruiksvriendelijkheid van de CMS
	\item Duur van het uitbouwen van de website
	\item Standaard performance van de website
	\item Aanpasbaarheid van de website
\end{itemize}[2]Multiplatform ondersteuning zal getest worden door middel van Browershots.org. In deze tool kan je verschillende browserversies selecteren, je krijgt meerdere screenshots van je website terug in alle browser types. Deze test zal eenmaal gedaan worden na de standaard implementatie van de website. Hierna zal er gekeken worden welke services er door het CMS-systeem aangeboden wordt om dit te verbeteren. Na de aanpassingen zal de test opnieuw afgenomen worden en zal er duidelijk te zien zijn  welke verschillen deze teweeg gebracht hebben.
Responsiveness zal getest worden met behulp van een online mobielvriendelijke test. Dit zal getest worden na de standaard implementatie. In het geval van slechte resultaten zal er betracht worden deze te  verbeteren met de aangeboden services binnen het CMS. De website zal dan opnieuw getest worden zodat er duidelijk wordt welk verschillen deze aanpassingen teweeg gebracht heeft.
Met aanpasbaarheid van de website wordt er bedoelt welke uitbreidingen er aangeboden worden. Dit wordt bij de andere factoren al uitgebreid getest. De verschillende uitbreidingen die beschikbaar zijn en gebruikt worden zullen uitbundig beschreven worden in de gemaakte documentatie. Er zal steeds vermeld worden waarom er niet voor die uitbreiding gekozen is.
Tijdens het maken van deze websites in de verschillende CMS'en zal alles goed gedocumenteerd worden. Er zal steeds duidelijk vermeld worden welke thema's, modules en  plug-ins er worden gebruikt. Voor deze aangeboden services binnenin de CMS'en zal er telkens geopteerd worden voor de gratis services. In deze documentatie zullen ook volgende zaken te lezen zijn: welke obstakels er waren, welke thema's er gebruikt zijn, welke modules er geprobeerd zijn, etc.
Het uiteindelijk antwoord op de verschillende onderzoeksvragen zal niet per se het CMS zijn met de beste scores op de verschillende factoren. Het besluit zal eerder gebaseerd zijn op de gebruiksvriendelijkheid van het systeem en de website. Het besluit zal ook losstaan van de technische achtergrond van de gebruiker en de grootte van het project.
Het antwoord op de SEO onderzoeksvraag zal onderzocht worden aan de hand van verschillende tools namelijk:

\begin{itemize}
	\item Siteliner: zoeken voor duplicate content.
	\item Wordtracker scout: keywords van concurrentiële websites te zien.
	\item Pingdom: Testen van de performance van de website.
	\item Google Trends: Bepalen van populaire keywords.
\end{itemize}
De onderzoeksvraag voor SEO zal enkel van toepassing zijn op de one-page portfolio website. De bovenvermelde tools zullen gebruikt worden voor het testen van de SEO-optimalisatie van de website na een standaard implementatie. Echter zullen deze tools ook gebruikt worden om de SEO-optimalisatie van de website te verbeteren. Het uiteindelijke antwoord op deze onderzoeksvraag zal niet enkel afhangen van de beste score na standaard implementatie maar ook welk CMS hier de makkelijkste aanpassingen in aanbiedt.
Voor de onderzoeksvraag over de e-commerce website zullen er e-commerce oplossingen toegevoegd worden in de website. In deze onderzoeksvraag zullen er ook enkele populaire e-commerce oplossingen met elkaar vergeleken worden zoals WooCommerce, shopify en Drupal Commerce \autocite{BWPL2017a}. Hier zal er gekeken worden welke invloed deze extensie hebben op de snelheid van de website. Is er SEO-optimalisatie aanwezig in de extensie? Welke moeilijkheden brengt deze extensie met zich mee? Op basis hiervan zullen de andere extensies uitgesloten worden. De overblijvende extensie zal effectief geïmplementeerd worden in de e-commerce website.

%---------- Verwachte resultaten ----------------------------------------------
\section{Verwachte resultaten}
\label{sec:verwachte_resultaten}

Er zullen zeker merkbare verschillen zijn in hoe goed de standaard implementatie van de verschillende CMS'en scoren op de verschillende factoren. De standaard implementaties van Drupal zullen beter scoren in performance en SEO dan die van de andere CMS'en. Echter wordt er wel verwacht dat deze verschillen makkelijk verkleind kunnen worden door de grote aanpasbaarheid van Wordpress en Joomla. Wordpress en Joomla staan in de e-business wereld nu eenmaal bekend voor het groot aanbod in plug-ins en modules. Er worden twee resultaten bijgehouden de score bij de standaard implementatie en de score na de aangebrachte modificaties. In de finale beslissing zullen de tegengekomen obstakels ook een factor spelen.

%---------- Verwachte conclusies ----------------------------------------------
\section{Verwachte conclusies}
\label{sec:verwachte_conclusies}

Voor de onderzoeksvraag in verband met SEO wordt er verwacht dat Drupal als winnaar uit de bus zal komen. Dit simpelweg door dat Drupal standaard een SEO-optimalisatie geïmplementeerd heeft en deze nog verder uitbreidbaar is met verschillende modules. Joomla zal als winnaar uit de bus komen voor de onderzoeksvraag in verband met de e-commerce website. Dit is gebaseerd op het feit dat Joomla bekend staat om zijn eenvoudige op te stellen e-commerce websites. De veronderstelling is dat Wordpress als 'grootste winnaar' uit de bus zal komen. Dit omdat er veronderstelt wordt dat Wordpress het antwoord zal zijn op twee verschillende onderzoeksvragen. Namelijk de onderzoeksvraag in verband met de blog en de one-page website. Deze veronderstelling is gebaseerd op het feit dat Wordpress hoofdzakelijk gecreëerd was voor bloggers.
